% !TEX root = DesignDocument.tex


\chapter{System  and Unit Testing}

This section describes the approach taken with regard to system and unit testing. 

\section{Overview}
Testing is done manually, via a tracibility matrix. Each requirement listed in the backlog will have a test case, and the system will either pass the test, or fail the test. 

\section{Test Setup and Execution}
Each major component has a list of requirements, and a test case that tests that the system meets the requirements. Some requirements only detail the presence of an element, and are tested by examination. The test case for these requirements will note that the test is by examination. 

Note: Manual Unit Testing - list which functions are tested by test case... maybe list by test case

Unit testing is an important process that tests individual functions or modules. Unit testing is often automated but it can also be done manually. I believe you have done some unit testing along development for verification, so you just need to document it.

\section{System Testing}

\subsection{Test Case \#1}
Select event set on index page, navigate to report page, verify that selection was maintained. Refresh page and verify that selection was maintained.
\subsubsection{Result}
Event set state was maintained.
\subsubsection{Code Coverage}
\begin{itemize}
\item file: function name
\item file: function name
\end{itemize}
\subsubsection{Requirements Fulfilled}
\begin{itemize}
\item Navigation Bar: The selected event set shall maintain its state when navigating to a different page or refreshing the page.
\end{itemize}

\section{System Integration Analysis}
The system is a standalone system and will not need to be integrated into another system. Other mapping systems, such as the USGS Did you Feel it (DYFI) project and the USGS Butterfles and Moths of North America (BAMONA), may need to be integrated with this system later. Inegration with these other systems will need to be done by hand.

How does this system get moved to the WWW?

\section{Risk Analysis}
If this project were up and running, what could go wrong? Server capabilities? Server crash? Database crash? Database hack? Accidentially drop tables?

From Dr. Qiao:
 Scalability is an common issue existing in mostly all systems and applications, because there’s always a limitation. I think the point here is not to tune your senior design project to handle a great amount of requests or traffic volume. Instead, you want to know the capacity of your current system, like memory, TCP connection, DB connection pool, etc., and have an outline on scale-up approaches. For this project, I think you may consider opcode cache for PHP code, on-the-fly cache for webpage, and memcache and load balancing for DB.  
 
Security is a non-functional requirement, but it’s an critical factor in the system. Some basic front-end (Javascript) and back-end (php) input validation and password encryption (hashed and salted) will make the system more secure and reliable for users. 

\subsection{Risk Mitigation}


