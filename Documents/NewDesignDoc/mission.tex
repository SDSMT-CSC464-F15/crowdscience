% !TEX root = DesignDocument.tex

\section{Mission Statement}
\subsection{Product Description}

Crowdsourcing is a method of gathering information from a large group of people, especially from internet users, rather than employing more traditional methods to complile information. Right now, there are several ongoing crowdsourcing projects, such as the USGS Did You Feel It (DYFI) project, USGS Butterflies and Moths of North America (BAMONA) project, and Cornell’s eBird project. Crowdsourcing reduces time and cost for data acquisition and enhances the accuracy and generality of research results.

The Crowd Science Mapper will be a generic crowdsourcing toolkit, easily adaptable to any kind of information being gathered. Ordinary citizens will be able to report information via an event reporting interface and view information with an event viewing interface. Academics and Researchers will be administrators in the system, and will not need programming experience. System administrators will be able to moderate event reports and create and modify sets of event reports through a graphical user interface. 

\subsection{Goals}
The goal of this project is to distribute a finalized toolkit or API to researchers and general public by 2018. The first part of this goal is to create a tool that is versatile for any type of data researchers want to research, easy and intuitive enough for any user to report events and view data, and reliable and secure enough for academic use. The second part of this goal will be to distribute that tool to a wide range of people to provide a wide range of data points for researchers using the tool to collect data. 

\section{Elevator Pitch}
Crowdsourcing aims at getting massive amounts of information from the online community. There are several individual corwdsourcing projects, collecting information about earthquakes, birds, butterflies, and more. The Crowd Science Mapper will be a generic crowdsourcing toolkit, that can accommodate any kind of information that researchers are gathering, and won't require any technicial know-how to use. 