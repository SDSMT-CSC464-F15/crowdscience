
% !TEX root = DesignDocument.tex

\chapter{Design  and Implementation}
This section is used to describe the design details for each of the major components 
in the system.    Note that this chapter is critical for all tracks.  Research tracks would do experimental design here where other tracks would include the engineering design aspects.    This section is not brief and requires the necessary detail that 
can be used by the reader to truly understand the architecture and implementation 
details without having to dig into the code.   

 \section{Architecture and System Design}
 This is where you will place the overall system design or the architecture.   This section should be image rich.  There is the old phrase {\it a picture is worth a thousand words}, in this class it could be worth a hundred points (well if you sum up over the entire team).   One needs to enter the design and why a particular design has been done.   
 
 \subsection{Design Selection}
 Failed designs, design ideas, rejected designs here.
 
 \subsection{Data Structures and Algorithms}
 Describe the special data structures and any special algorithms.
 
 \subsection{Data Flow}
 
 \subsection{Communications}
 
 \subsection{Classes}
 
 \subsection{UML}
 
 \subsection{GUI}
 
 \subsection{MVVM, etc}

\section{Major Component \#1 }

\subsection{Technologies  Used}
This section provides a list of technologies used for this component.  The details 
for the technologies have already been provided in the Overview section. 

\subsection{Component  Overview}
This section can take the form of a list of features. 

\subsection{Phase Overview}
This is an extension of the Phase Overview above, but specific to this component. 
 It is meant to be basically a brief list with space for marking the phase status. 

\subsection{ Architecture  Diagram}
It is important to build and maintain an architecture diagram.  However, it may 
be that a component is best described visually with a data flow diagram. 


\subsection{Data Flow Diagram}
It is important to build and maintain a data flow diagram.  However, it may be 
that a component is best described visually with an architecture diagram. 


\subsection{Design Details}
This is where the details are presented and may contain subsections.   Here is an example code listing:
\begin{lstlisting}
#include <stdio.h>
#define N 10
/* Block
 * comment */
 
int main()
{
    int i;
 
    // Line comment.
    puts("Hello world!");
 
    for (i = 0; i < N; i++)
    {
        puts("LaTeX is also great for programmers!");
    }
 
    return 0;
}
\end{lstlisting}
This code listing is not floating or automatically numbered.  If you want auto-numbering, but it in the algorithm environment (not algorithmic however) shown above.
