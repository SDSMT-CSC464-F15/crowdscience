
% !TEX root = DesignDocument.tex

\chapter{Design  and Implementation}
This section contains the architecture and implementation details for each of the major components in the system.

 \section{Architecture and System Design}
 
 \subsection{Data Flow} 
 In general data flows from the HTML form to the associated JavaScript, which posts requests to PHP files to access data in the Database. See Figure~\ref{design_dataflow}.

\begin{figure}[tbh]
\begin{center}
\includegraphics[width=0.75\textwidth]{./figures/design_dataflow.png}
\end{center}
\caption{System Architecture and Design Overview\label{design_dataflow}}
\end{figure}
 
 \subsection{Communications} Data is communicated to and from the user via HTML. Data is communicated to and from the database via PHP and JavaScript to the HTML that the user sees. See Figure~\ref{design_dataflow}.
 
 \subsection{MVC} The project has a Model View Controller architecture. The model primarily consists of PHP files, which access the underlying data in the database. The view is the HTML files, which contain only page layout information. The controller is in the JavaScript, which controls dataflow to and from the view (HTML) and model (PHP) layer. See Figure~\ref{design_dataflow}.
 
 \subsection{GUI} See Figure.
 
\section{Major Component \#1 }

\subsection{Technologies  Used}
This section provides a list of technologies used for this component.  The details 
for the technologies have already been provided in the Overview section. 

\subsection{Component  Overview}
This section can take the form of a list of features. 

\subsection{Phase Overview}
This is an extension of the Phase Overview above, but specific to this component. 
 It is meant to be basically a brief list with space for marking the phase status. 

\subsection{ Architecture  Diagram}
It is important to build and maintain an architecture diagram.  However, it may 
be that a component is best described visually with a data flow diagram. 


\subsection{Data Flow Diagram}
It is important to build and maintain a data flow diagram.  However, it may be 
that a component is best described visually with an architecture diagram. 


\subsection{Design Details}
This is where the details are presented and may contain subsections.   Here is an example code listing:
\begin{lstlisting}
#include <stdio.h>
#define N 10
/* Block
 * comment */
 
int main()
{
    int i;
 
    // Line comment.
    puts("Hello world!");
 
    for (i = 0; i < N; i++)
    {
        puts("LaTeX is also great for programmers!");
    }
 
    return 0;
}
\end{lstlisting}
This code listing is not floating or automatically numbered.  If you want auto-numbering, but it in the algorithm environment (not algorithmic however) shown above.
