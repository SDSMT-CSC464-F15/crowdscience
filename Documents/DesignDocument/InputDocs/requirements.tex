% !TEX root = ../SystemTemplate.tex
\chapter{User Stories, Backlog and Requirements}
\section{Overview}

This section contains basic requirements for the Crowd Science Mapper project. The Crowd Science Mapper will be a generic crowdsourcing website, where ordinary citizens can report events such as butterfly or bird sightings, and view these events. Administrators in this system will be able to approve individual events and modify event sets, with no programming experience using a graphical user interface.

\subsection{Scope}

This section contains stakeholder information, initial user stories, requirements, product backlog, and proof of concept results.

\subsection{Purpose of the System}
The purpose of this crowdsourcing website is to provide a customizable interface for researchers and academics to collect data about scientific events from ordinary citizens. 

\section{ Stakeholder Information}

Stakeholders for this project are academics and researchers who would eventually be the administrators of this project, and ordinary citizens who would be the main users, in the feild reporting events. 

\subsection{Customer or End User (Product Owner)}

Product owners are Dr. Mengyu Qiao and Gail Schmidt, who will assist in the project in mentor roles. As needed, they will assist with prioritizing the product backlog, and identify important features. Intellicetual property in this project belongs to South Dakota School of Mines and Technology. 

\subsection{Developers --Testers}
Developers and testers are Jiasong Yan and Hannah Aker. Testers will eventually include SDSMT faculty and students, with faculty as administrators and students as ordinary users reporting events.

\section{Requirements and Design Constraints}

This project will be a HTML based webpage using Java Script to connect to a Mongo database, and hosted on an Apache server.

\subsection{System  Requirements}
Webpage should be able to run on all operating systems and all internet browsers. This project is not required to have mobile functionality.

\subsection{Network Requirements}
This project will be server based, hosted on an Apache 2.0 server, access supplied to us by Gail Schmidt. The server is a cloud based server operated by SGT. Information about Apache 2.0 servers can be found here: \url{http://httpd.apache.org/docs/2.0/}.

\subsection{Development Environment Requirements}
Development will be done in HTML, CSS, PHP and Java Script. The project will connect to a Mongo database. Information about HTML, CSS, PHP and Javascript can be found here: \url{http://html.net/}.

\subsection{Project  Management Methodology}
Source control will be Github; github issues will be used to keep track of backlog and sprint status. The github repository is located at: \url{https://github.com/SDSMT-CSC464-F15/crowdscience} All parties have access to the Sprint and Product Backlogs. There will be 6 sprints total in this project, each lasting 3 weeks.


\section{User Stories}

\subsection{User Story \#1}
As a user, I want all the core functionality of the landscape change mapper to be maintained. These functions include, but are not limited to:
\begin{itemize}
\item Visual map representation of events
\item Detailed event list
\item Event reporting interface
\item New user registration
\item User login
\end{itemize}

\subsection{User Story \#2} 
As an administrator, I want to be able to customize a set of events to fit my needs. This may include user input and display items, database design, digital map, webpage color and style, logo, etc. 

\subsection{User Story \#3} 
As an administrator, I want to be able to edit existing event reports.

\subsection{User Story \#4} 
As a user, I want to be able to select which set of events I would like to view.

\section{Requirements}
The requirements are directly derived from the user stories, but much more refined. These requirements will map directly to test cases in a tracibility matrix.

\subsection{Global}
\begin{itemize}
\item Each page shall contain a navigation bar at the top of the page.
\item The navigation bar will have buttons for logging in, new user registration, administration, and event reporting.
\item Each page shall contain a drop down box for selecting the event set.
\end{itemize}

\subsection{New User Registration}
\begin{itemize}
\item When no user is logged in, there will be a button on the main page leading to a new user registration page.
\item The registration page will have a feild for username, password and password verification, and a ``submit'' button.
\item If the passwords do not match, the user will be notified, and need to reenter one or both passwords.
\item After registering, a user will automatically be logged in.
\end{itemize}

\subsection{User Login}
\begin{itemize}
\item When no user is logged in, there will be a button on the main page leading to a login page
\item The login page will have fields for entering username and passwords, a ``login'' button and a link to the user registration page.
\item The user will be notified if the username or password was incorrect, and may need to reenter one or both feilds.
\item When the user is logged in, an icon indicating the logged in user and a log out button will replace the login button on the navigation page.
\end{itemize}

\subsection{Event reporting interface}
\begin{itemize}
\item When a user is logged in, the user is listed as the author of the report.
\item When no user is logged in, the author of the report is ``Annonymous''.
\item Every event report shall include a longitude and latitude feild.
\item The event reporting window shall contain the feilds specified for that event data set.
\item Each required feild will be marked as required.
\item The event reporting window shall contain a submit button.
\item The user will be notified if they have not entered all required feilds, and will be able to return to edit their unfinished report.
\end{itemize}

\subsection{Detailed event list}
\begin{itemize}
\item The detailed event list will be a table of the current event set, and will include all feilds specified for that event data set.
\item Each entry shall include longitude, latitude of event, and the author of the event.
\item When a point on the map is hovered over, the associated entry in the list will be highlighted.
\item When a point on the map is clicked, the associated entry in the list will be highlighted in a different shade.
\end{itemize}

\subsection{Map representation of events}
\begin{itemize}
\item The map shall be a visual representation of the event data set. 
\item The map shall contain a marker for each event report, placed at the longitude and latitude specified in the report.
\item The map will be able to be zoomed in, zoomed out, and panned.
\item Entries in the event list will be highlighted when a marker is hovered over or clicked.
\end{itemize}

\subsection{Administrator Login}
\begin{itemize}
\item When an administrator logs in, they wil lbe able to see buttons in the navigation bar that lead to pages for editing events and customizing event sets.
\end{itemize}

\subsection{Administrator editing of existing events}
\begin{itemize}
\item The detailed data list shall be displayed in a seperate window.
\item Each event report in the list will have an ``edit'' button.
\item When an administrator clicks the ``edit'' button, they will be taken to the event reporting interface, with the added feild ``Reason for editing''.
\item When an administrator edits an event report, the administrator name, reason for editing, and date and time shall be recorded and added to the event's history.
\item An administrator may mark an event report as false, the administrator name, reason for declaring false, and date and time shall be added to the event's history.
\end{itemize}

\subsection{Administrator customization of event sets}
\begin{itemize}
\item The administrator will be able to create a new event data set.
\item The administrator will be able to edit the details of an existing event data set.
\item When the administrator clicks on the new event data set, a blank event data set will be created. 
\item The display name and identification name of an event data set cannot be empty or null.
\item The identification name will be used to reference the database, and cannot be changed after creation.
\item The administrator will be able to add feilds to their event data set by clicking an ``Add new Feild'' button.
\item The administrator will be able to designate the feild identification name, display name, data type, display method if appliciable, and acceptable values if appliciable. 
\item If the administrator wants to change the data type of the feild, they will need to remove the current feild and recreate it with the new data type, and data stored in the previous type will be lost.
\item The administrator will be able to edit display name, display method, and acceptable values with ease.
\item The data types that can be selected will be number, short text, long text, radio button group, check box group, and picture upload.
\item When the administrator is finished editing or creating their event data set, they will be able to click ``save'' to save their data. 
\item If there are problems with what the administrator has entered, the administrator will be notified and prompted to fix the errors.
\end{itemize}

\subsection{User selection of data to view}
\begin{itemize}
\item User will beable to select from a drop down menu on any page which event data set to view. 
\item When the current event data set is changed, the map, event list, administrator functions and report features will change according to the feilds specified for that set.
\item If implemented, the general appearance of the pages will change according to the selected event data set.
\end{itemize}

