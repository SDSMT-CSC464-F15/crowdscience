% !TEX root = DesignDocument.tex


\chapter{Overview and concept of operations}

The overview should take the form of an executive summary.  Give the reader a feel 
for the purpose of the document, what is contained in the document, and an idea 
of the purpose for the system or product. 

\section{Team Members and Team Name}


\section{Client}
A description of the client or customer.

\section{Project}
A high level description of the project.

\subsection{Purpose of the System}
What is the purpose of the system or product? 


\section{Business Need}
Use this section to define what business need exist and how this software will 
meet and/or exceed that business need.   


\section{Deliverables}

Provide a complete description of the client requested deliverables.   This section should be the section your software contract references.   

\section{System Description}

\subsection{Major System Component \#1}
Describe briefly the role this major component plays in this system. 

\subsection{Major System Component \#2}
Describe briefly the role this major component plays in this system. 

\subsection{Major System Component \#3}
Describe briefly the role this major component plays in this system. 

\section{Systems Goals}
Briefly describe the overall goals this system plans to achieve.
These goals are typically provided by the stakeholders.  This is not
intended to be a detailed requirements listing.  Keep in mind that
this section is still part of the Overview.

\section{System Overview and Diagram}
Provide a more detailed description of the major system components
without getting too detailed.  This section should contain a
high-level block and/or flow diagram of the system highlighting the
major components. weill try to put it close to where it was
typeset but will not allow the figure to be split if moving it can not
happen.  Figures, tables, algorithms and many other floating
environments are automatically numbered and placed in the appropriate
type of table of contents.  You can move these and the numbers will
update correctly.


\section{Technologies Overview}
This section should contain a list of specific technologies used to
develop the system.  The list should contain the name of the
technology, brief description, link to reference material for further
understanding, and briefly how/where/why it was used in the system.
See Table~\ref{somenumbers}.  This is a floating table environment.
\LaTeX\ will try to put it close to where it was typeset but will not
allow the table to be split.

\begin{table}[tbh]
\caption{A sample Table ... some numbers. \label{somenumbers}}
\begin{center}
\begin{tabular}{|r|l|}
  \hline
  7C0 & hexadecimal \\
  3700 & octal \\ \cline{2-2}
  11111000000 & binary \\
  \hline \hline
  1984 & decimal \\
  \hline
\end{tabular}
\end{center}
\end{table}

