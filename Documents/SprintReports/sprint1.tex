% !TEX root = sprints_wrapper.tex

\section{Team Members}
Hannah Aker and Jiasong Yan
\section{Project Sponsors}
Dr. Mengyu Qiao and Gail Schmidt
\section{Sponsor/User Description}
\subsection{User Description}
Primary users will be the everyday citizen, interested in reporting some event, such as butterfly sighting, geological or landscape changes, etc. Secondary users will be academics and researchers who will use the gathered information in their research, they will be administrators of this data.
\subsection{Project Goal}
The goal of this project to improve on the idea originally presented in the Landscape Change Mapper, and expand on it to create a flexible interface for other kinds of events.
\subsection{User Needs}
Primary users need all the core fonctionality of the Landscape Change Mapper to be maintained.These functions include, but are not limited to:
\begin{itemize}
\item Visual map representation of events
\item Detailed event list
\item Event reporting interface
\item New user registration
\item User login
\end{itemize}
Users also need to be able to select which set of events they would like to view.
Administrators need to be able to customize a set of events to fit my needs. This may include user input and display items, database design, digital map, webpage color and style, logo, etc. Administrators will also need to be able to edit existing event reports.
\section{Project Overview}
The Crowd Science Mapper will be a generic crowdsourcing system framework and toolkits, which can be customized by ordinary users with no programming experience using graphical user interface
\section{Project Environment}
\subsection{Project Boundaries}
While the previous project included a mobile application, this project will not because the team is smaller. This project will be solely web-based. 
\subsection{Project Context}
This project will use the same general context that the Landscape Change Mapper used.  
\subsubsection{Technical Environment}
This project will use the same enviroment used in the Landscape Change Mapper. This will use HTML, PHP and Java Script to connect to a Mongo Database, and was hosted on an Apache 2.2 Server.
\subsubsection{Current systems overview}
The Landscape Change Mapper webpages used HTML, PHP and Java Script to connect to a Mongo Database, and was hosted on an Apache 2.2 Server. The webpage contained a map of events, detailed event list, event reporting interface, user login, and new user registration. The project included a mobile app with an event reporting interface. 
\section{Project Deliverables}
The project deliverable will be a proof-of-concept webpage with the features listed in the backlog.
\section{Backlog}
The following features need to be added to this project:
\begin{itemize}
\item Visual map representation of events
\item Detailed event list
\item Event reporting interface
\item New user registration
\item User login
\item User selection of data set to view
\item Administrator login
\item Administrator customization of event sets, including user input and display items, database design, digital map, webpage color and style, logo, etc. 
\item Administrator editing of existing event reports
\end{itemize}
\section{Potential Issues}
Potential issues might stem from using Java Script to interface with the Mongo DB. We don't have much prior experience with these specific tools, though we have used similar tools.